\documentclass[a4paper,12pt]{article}
\usepackage{graphicx}
\usepackage{algorithmicx}
\usepackage{algorithm}
\usepackage{algorithmic}

\author{E. Routledge}
\date{01 Nov 2022}
\title{Sudoku and Other Related Problems}

\begin{document}
\maketitle

\section{Sudoku is hard}

Sudoku is a relatively simple game, in the standard 9 by 9 (or 3 by 3 by 3 by 3) one must complete the grid such that every row column and box contains the numbers 1 to 9, that is all. So, yes, sudoku is simple in the sense it can be described in a single rule but it is in fact hard. Instances of the puzzle requiring complex x wing and y wing strategies are not what makes the puzzle hard, it is hardness through a provable, mathematical (computational) lense for which this paper cares. 

\subsection{Finding a solution is hard}
The standard way to solve a 9 by 9 sudoku puzzle is by the backtracking algorithm. One learns the idea of recursion through these toy examples.

\begin{algorithm}
	\caption{Backtracking(grid, dim)}\label{backtracking}
	\begin{algorithmic}[1]
		\For{next empty square}
			\For{i = 1 to dim}
				\If{value i in empty square is valid}
					\If{Backtracking(grid with i in empty square, dim) returns valid solution}
						\State grid \leftarrow Backtracking(grid with i in empty square, dim)
				\EndIf
			\If{no value is valid} \Return False \Endif
			\Endfor
		\Endfor
		\If{no empty squares} \Return grid \Endif
	\end{algorithmic}
\end{algorithm}

\subsection{Uniqueness is hard?}
It is hard to determine if a puzzle has a unique solution?

- generate uniqueness


\end{document}
\vspace{5mm}