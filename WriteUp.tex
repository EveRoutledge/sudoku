\documentclass[a4paper,12pt]{article}
\usepackage{listings}
\usepackage{amsmath}

\author{E. Routledge}
\date{01 Nov 2022}
\title{Sudoku and Other Related Problems}

\begin{document}
\lstset{language=Python}
\maketitle

% ~~~~~~~~~~~~~~~~~~~~~~~~~~~~~~~~~~~~~~~~~~~~~~~~~~~~~~~~~~~~~~~~~~~~~~~~~~~~~~~~~~~~~~~~~~~~ %
\section{Introduction}
		Sudoku is a simple logic game, in the standard $9 \times 9$ (or $3 \times 3 \times 3 \times 3$) 
		one must complete the grid such that every row, column and box contains the numbers 1 to 9, that is all. 
		Yet it leads to interesting and difficult puzzles.
	\subsection{History}
% ~~~~~~~~~~~~~~~~~~~~~~~~~~~~~~~~~~~~~~~~~~~~~~~~~~~~~~~~~~~~~~~~~~~~~~~~~~~~~~~~~~~~~~~~~~~~ %
\section{Sudoku is Hard}

		Sudoku can be described in a single rule but it is in fact a hard problem to solve.
		Instances of the puzzle requiring complex x wing and y wing strategies are not what makes the puzzle hard to solve,
		it is hardness through a provable, mathematical (computational) lense for which this paper cares. 
		
	\subsection{Finding a Solution is Hard}
		
		Finding a solution to sudoku is NP-complete that is to say it is computationally hard and no known algorithm can solve this in resonable time (polynomial). 

		\begin{equation}
		        \Phi (\text{grid}) = \begin{cases}
		            \text{True if a solution exists} \\
		            \text{False if a solution does not exist}
				\end{cases}
		\end{equation}

		Proof
		
		We do this through a reduction to a known np-complete problem.
		Assume we have an oracle $\Phi$ that when given an instance of the problem (in this case a grid of numbers and blanks) will,
		in polynomial time or less return True if it can be solved and False otherwise.

	\subsection{Determining Uniqueness is Hard}

		It is hard to determine if a puzzle has a unique solution?
		
	\subsection{Validation is Easy}
% ~~~~~~~~~~~~~~~~~~~~~~~~~~~~~~~~~~~~~~~~~~~~~~~~~~~~~~~~~~~~~~~~~~~~~~~~~~~~~~~~~~~~~~~~~~~~ %
\section{Other Related Problems}
	\subsection{Latin Squares}

		A latin square is an n by n matrix filled with n characters that must not repeat along columns or rows.
		Reduced Form -  if first row and column is in the natural order
		Equivalence classes
		Number of n by n latin squares is bounded
		Latin squares can be considered a bipartite graph
		Agronomic Research
		Latin hypercube

	\subsection{Magic Squares}

		A magic square is a matrix of numbers with each column, row and diagonal summing to the same value, 
		this value is known as a magic constant and the degree is the number of columns/rows.
		A normal magic square is one containing the integers 1 to $n^2$.
		Magic Squares with repeating digits are considered trivial.
		Semimagic squares omit the diagnonal sums also summing to the magic constant.

		Truly thought to be magic Shams Al-ma'arif.

		Generation, there exists not completely general techniques. Diamond Method

		Associative Magic Squares
		Pandiagonal Magic Squares
		Most-Perfect Magic Squares
		
		Equivalence classes for $n<=5$ but not for higher orders.
		The enumeration of most perfect magic squares of any order.

		880 distinct magic squares of order four

		Properties
			Normal magic squares can be constructed for all values except 2
			Preserving the magic property when transformed

		Methods of construction

		Multiplicative magic squares - produce infinite

		Sator square

		magic square of squares - Parker Square is a failed example of this
	
	\subsection{Greco-Latin Squares}

		Two orthogonal latin squares super imposed, such that the pairs of values are unique.
		Group based greco latin squares
		Eulers interest came from construction of magic squares
		

% ~~~~~~~~~~~~~~~~~~~~~~~~~~~~~~~~~~~~~~~~~~~~~~~~~~~~~~~~~~~~~~~~~~~~~~~~~~~~~~~~~~~~~~~~~~~~ %
\section{Solving Techniques}
	\subsection{Backtracking}
		
		The standard way to solve a $9 \times 9$ sudoku puzzle is by the backtracking algorithm. 
		This is a brute force method with a few optimisations.
		One can expect to find this algorithm in a computer science course introduction to recursion, that is to say it is not a complex concept 
		and while useful for the usual sizes, as soon as we increase to $16 \times 16$ this becomes infeesible.
		Multiplication tables of quasigroups.
		Orthogonal latin squares are used in error correcting codes.
		
		\begin{lstlisting}[caption=Backtracking]
def Backtracking(grid):
    for each row:
        for each column:
            if grid is empty at this potion:
                try a value in this position
                Backtracking(grid with new value)
                if successful:
                    return grid
                else:
                    try another value
                if no values left to try:
                    return False
    return grid						
		\end{lstlisting}
		
		Why does brute force not work for larger examples?

	\subsection{Stochastic Methods}
		\subsubsection{Simulated Annealing}
		\subsubsection{Genetic Algorithm}
% ~~~~~~~~~~~~~~~~~~~~~~~~~~~~~~~~~~~~~~~~~~~~~~~~~~~~~~~~~~~~~~~~~~~~~~~~~~~~~~~~~~~~~~~~~~~~ %
\section{Generating  Techniques}
% ~~~~~~~~~~~~~~~~~~~~~~~~~~~~~~~~~~~~~~~~~~~~~~~~~~~~~~~~~~~~~~~~~~~~~~~~~~~~~~~~~~~~~~~~~~~~ %
\section{17 is the Magic Number}
	\subsection{Sparsity - information theory}
		Bomb sudoku/latin squares - Additional rule: the same number can not occur in adjacent or diagonally adjacent squares .
% ~~~~~~~~~~~~~~~~~~~~~~~~~~~~~~~~~~~~~~~~~~~~~~~~~~~~~~~~~~~~~~~~~~~~~~~~~~~~~~~~~~~~~~~~~~~~ %
\section{Group theory}
	\subsection{Equivalence Classes}
% ~~~~~~~~~~~~~~~~~~~~~~~~~~~~~~~~~~~~~~~~~~~~~~~~~~~~~~~~~~~~~~~~~~~~~~~~~~~~~~~~~~~~~~~~~~~~ %
\section{Topology}
	\subsection{Torus}
% ~~~~~~~~~~~~~~~~~~~~~~~~~~~~~~~~~~~~~~~~~~~~~~~~~~~~~~~~~~~~~~~~~~~~~~~~~~~~~~~~~~~~~~~~~~~~ %

\begin{thebibliography}{100}
	\bibitem{LatinSquaresAlgorithm} https://onlinelibrary.wiley.com/doi/10.1002/(SICI)1520-6610(1996)4:6<405::AID-JCD3>3.0.CO;2-J
	\bibitem{LatinSquareAgronomicResearch} http://joas.agrif.bg.ac.rs/archive/article/59
	\bibitem{LatinSquareCommunication} https://www.semanticscholar.org/paper/Permutation-arrays-for-powerline-communication-and-Colbourn-Kløve/7e69cfdbd2082463c66de698da1e326f0556d1d4
	\bibitem{MagicSquares1} http://www.multimagie.com/English/SquaresOfSquaresSearch.htm
	\bibitem{MagicSquaresRelationToSudoku} https://plus.maths.org/content/anything-square-magic-squares-sudoku
\end{thebibliography}


\end{document}
\vspace{5mm}